\chapter{Technologien}
Die Webanwendung \emph{Hurace.Web} wurde wie gefordert mit dem SPA-Framework \emph{Angular} (in der Version 8.2) entwickelt.
Um die Entwicklung zu vereinfachen wurden außerdem folgende externe Bibliotheken verwendet:

\section{Angular Material}
\emph{Angular Material}\footnote{\url{https://material.angular.io/}} bietet Angular-Komponenten (\zB Formularelemente, Navigationsleiste, Lade-Anzeige, ...) im Material-Design von Google an.
Da auch die Desktopanwendung von Hurace im Material-Design gestaltet wurde, ergibt die erneute Verwendung in der Webanwendung ein einheitliches Design für den Benutzer.

\section{ngxs}
\emph{ngxs}\footnote{\url{https://www.ngxs.io/}} ist eine Zustandsverwaltungs-Bibliothek die nach dem CQRS-Muster entworfen und ähnlich zu \emph{Redux}\footnote{\url{https://redux.js.org/}} bzw. \emph{ngrx}\footnote{\url{https://ngrx.io/}} aufgebaut ist.
Durch den Einsatz von \emph{ngxs} kann die Logik zum Laden oder Verändern von Daten (anstatt in den Komponenten) in einen (bzw. mehrere) Service ausgelagert werden, der den Applikationszustand managet.
Da der Applikationszustand nur zentral verändert werden kann können Zustandsveränderungen einfacher verstanden und analysiert werden.

\section{Auth0}
\emph{Auth0}\footnote{\url{https://auth0.com/}} kümmert sich um das Authentifizieren in der Webanwendung.

\section{SignalR}
\emph{SignalR}\footnote{\url{https://dotnet.microsoft.com/apps/aspnet/signalr}} wird eingesetzt damit Echtzeitdaten eines Liverennens, wie aktuelle LäuferIn und Zwischenzeiten, vom ASP.NET-Core-Server an die Webanwendung übertragen werden können.
Dadurch muss die Webanwendung nicht alle \emph{n} Sekunden nachfragen, ob es schon neue Zwischenzeiten für ein Rennen gibt, sondern wird automatisch benachrichtigt.
